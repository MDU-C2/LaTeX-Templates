% LATEX TEMPLATE BY EMIL PERSSON
% Layout design: Bo Tonnquist, Baseline Management AB, 2018  
\documentclass[10pt]{projectdoc}
\fancyFoot{© ProjectBase 7.0 Baseline Management AB, 2018}{\today}
% \usepackage[style=ieee]{biblatex}
% \addbibresource{references.bib}
\title{Project plan}
\projectname{}
\clientname{}
\managername{}
\begin{document}
\maketitle
\thispagestyle{fancy}
\infotable

% Copy and Paste any table into https://www.tablesgenerator.com/
% File -> From latex code... | Edit the imported table and generate new

\section{Executive summary}
\helper{A short summary of the project plan}

\section{Mission}
\subsection{Background}
\helper{Description with a clear and defined connection to the goals. It is advisable
to connect to any related project in the background description.}
\subsection{Purpose}
\helper{The impact the project is expected to create, i.e. why it is important to execute the project.  }
\subsection{Goal}
\helper{The result the project should deliver, i.e. what should be achieved when the project is executed.  }
\subsection{Scope}
\helper{What is included as part of the project and must be performed in order to deliver the goal. The scope is described with a
WBS at the overarching level – main packages with a brief description of each. The complete WBS should be included as
an attachment.}
\subsection{Limitations}
\helper{What the project should not deliver. The purpose is to avoid false expectations among the different stakeholders.}

\section{Requirement specification}
\subsection{Product specification}
\helper{Requirements on the project’s result/product}
\subsection{Project specification}
\helper{Requirements on the execution and prioritization between the project’s triple constraints.}
\subsection{Prerequisites}
\helper{Demands on the project’s sponsor/owner or client that have to be achieved to ensure the project’s execution and result.}

\section{Handover \& Implementation}
\helper{How to deliver the product to the client and implement it into the environment it is meant for.}

\section{Situational analysis and stakeholders}
\subsection{SWOT-analysis}
\helper{Mapping and analysis of external and internal factors that might affect execution.}

\begin{table}[H]
    \begin{tabularx}{\columnwidth}{|X|X|X|X|X|}
\hline
Strengths & Weaknesses & Opportunities & Threats & Conclusions \\ \hline
          &            &               &         &             \\ \hline
          &            &               &         &             \\ \hline
          &            &               &          &             \\ \hline
          &            &               &         &             \\ \hline
\end{tabularx}
\end{table}
\subsection{Stakeholder mapping}
\helper{Mapping and analysis of individuals, groups and organizations that might affect the project or will be affected by the
project.}

\section{Planning}
\subsection{Milestone plan}
\helper{Stakeholders may want an overarching flow chart or table of the project’s most important milestones as a indicator if the project is falling behind.}
\subsection{Activity list}
\helper{List of activities where time and resources are estimated}

\begin{table}[H]
    \begin{tabularx}{\columnwidth}{|X|X|X|X|X|X|}\hline
        ID & Activity & Resources & Start & Stop \\ \hline
        &          &           &       &         \\ \hline
        &          &           &       &         \\ \hline
        &          &           &       &         \\ \hline
        &          &           &       &         \\ \hline
        &          &           &       &         \\ \hline
        &          &           &       &         \\ \hline
        &          &           &       &         \\ \hline
        &          &           &       &         \\ \hline
        &          &           &       &         \\ \hline
    \end{tabularx}
    \end{table}

\subsection{Sprint plan}
\helper{Alternatively to activity list, a project can be divided into sprints. 
The first sprint should be planned in detailed here but also determine sprint lenghts and frequency of sprint planning meetings}


\subsection{Schedule}
\helper{Activity plan with a time axis where duration and connection between activities and milestones are shown. Present the
schedule in a separate document. (If sprints are used, list sprints above in sprint plan)}

\section{Schedule \& staffing}
\subsection{Roles, responsibilities and authorities}
\helper{Organizational structure that specifies project roles and with this authorities and responsibilities.}
\subsection{Staffing plan}
\helper{Who is given which role in the project}

\section{Project budget}
\helper{The project’s preliminary calculation – a  outline of internal and external costs for resources needed to execute the
project.}

\begin{table}[H]
    \begin{tabularx}{\columnwidth}{|X|X|X|X|}
        \hline
        Internal costs & External costs & Other costs & Summary \\ \hline
                       &                &             &         \\ \hline
                       &                &             &         \\ \hline
                       &                &             &         \\ \hline
\end{tabularx}
\end{table}

\section{Communication and quality assurance}
\subsection{Reports and documents}
\helper{Rules and routines on how to follow up and report on the project.}
\subsection{Communication plan }
\helper{Plan for spreading information in the purpose of guaranteeing the right target group gets the right information at the
right time and through the right channels}


\begin{table}[H]
\begin{tabularx}{\columnwidth}{|X|X|X|X|X|X|}
    \hline
    Who & Why & What & When & How & Responsible                   \\ \hline
    &     &      &      &     &                                   \\ \hline 
    &     &      &      &     &                                   \\ \hline 
    &     &      &      &     &                                   \\ \hline 
    &     &      &      &     &                                   \\ \hline 
    &     &      &      &     &                                   \\ \hline
\end{tabularx}
\end{table}

\section{Risk analysis and response planning}
\helper{Risk identification, risk evaluation and risk Response Planning.}
\begin{table}[H]
\begin{tabularx}{\columnwidth}{|X|X|X|X|X|X|}
\hline
Risk & Probability(1 to 5) & Impact(1 to 5) & Risk(P*I) & Risk Response \\ \hline
     &                     &                &           &               \\ \hline
     &                     &                &           &               \\ \hline
     &                     &                &           &               \\ \hline
     &                     &                &           &               \\ \hline
     &                     &                &           &               \\ \hline
\end{tabularx}
\end{table}

\section{Other}
\helper{Anything else?}

% \printbibliography
% \appendices
\end{document}
