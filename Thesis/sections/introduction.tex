\section{Introduction}
\label{sec:intro}

The current sectioning is an example to illustrate the use of the Latex template to write the thesis report. The text for the sections and subsections should be adapted to reflect the content of each section best. For example, the "problem formulation" does not need to be a subsection of the introduction. It can be a complete
the section on its own (if there is enough material). 

The introduction can be seen as an expanded version of the abstract. The authors can have roughly the same structure but one or two paragraphs for each point in the summary. The following should be included:
\begin{itemize}
\item[--] Presentation of the field and topic of the work. This should come early and should capture interest. It can include a brief background and possibly important definitions of terms
\item[--] You can briefly describe the intended audience for the report. Whom have you written for?
\item[--] Brief overview of previous work and its limitations 
\item[--] Presentation of the assignment including purpose and research questions
\item[--] Description of your approach to the task, methodology and why it is appropriate
\item[--] Motivation: why the task is interesting, what the relevant questions are, why your approach is good and why the results are essential.
\item[--] description of the main results and their limitations and what is new in your work
\item[--] overview of the report
\end{itemize}

You can discuss the significance of the conclusions, but the introduction should only briefly summarise the results. No specialized terminology or mathematics should be included here. 

The introduction can be written as a funnel: area - sub-area - task - any sub-task - purpose. You then lead the reader towards a progressively more detailed and specific understanding of the task and purpose. By the end of the introduction, you and the reader should have a base of shared understanding. The reader should understand the task, the scope of the work, the methodology and its main contribution, i.e. what is new in your work. 

The other sections of the report may also need a short introduction at the beginning so that the reader understands the purpose of each section and its place in the report.


\subsection{Problem Formulation} 

In this section, you formulate and specify the three essential things: purpose, question and motivation. First, you should present the task at a high level and in detail and discuss why it is essential. Next, explain the assumptions and limitations. You can then formulate the aim and question from the description of the task. Keep in mind that once the purpose is met, the question should be able to be answered. It is also essential that the purpose and motivation are linked. Once the purpose and question are clear, you can start developing the objectives which must be achieved to reach the purpose. Each objective should be small, achievable and possible to evaluate.  


\emph{Tips!} Write down your research question on paper and put it next to the screen. This will help you remember the research question when working on the report.

This is how to use the references~\cite{Berndtsson607210, Blomkvist2014} or~\citet{Turing1950}.